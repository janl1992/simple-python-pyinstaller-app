%!TEX root = ../report.tex

\begin{document}
    \chapter{Equation checking using finite random variables}
    \label{bayestheoremfiniterandomvariables}
    \begin{figure}[H]
        \tikzstyle{my help lines}=[red]
\begin{center}
    \begin{tikzpicture}
        \draw [style=my help lines] (0,0) grid (5,3);
        \filldraw (3.5,1.5) node [] {$n_{ij}$};
        \filldraw (3.5,-0.5) node [] {$x_{i}$};
        \filldraw (-0.5,1.5) node [] {$y_{i}$};
        \filldraw (5.3,1.5) node [] {\bigg \} };
        \filldraw (5.7,1.5) node [] {$r_{j}$};
        %    \draw [decorate, decoration={brace, amplitude=10pt}, rotate=180](0.5,0.5) -- (0.5,5.0);
        \filldraw (3.5,3.3) node [] {\rotatebox{90}{\bigg \}}};
        \filldraw (3.5,3.6) node [] {$c_{i}$};
        %    \rotatebox{270}{B}C
        %    \draw (2,0)
        %    grid +(2,2);
    \end{tikzpicture}
\end{center}
        \caption{Probability inside a grid}
        \caption*{Fig.~\textit{source} \cite[p.13]{bishop2006pattern} Here the probability for two random variables can be seen. The number of points in row
        $y$ illustrates the probability of $Y=y_{j}$. The number of points in column $i$ correspond to the probability of $X=x_{i}$.} \label{fig:Probability for two random variables}
    \end{figure}
    The idea of this approach is to search inside a string for patterns that match predefined axioms. These axioms are used as a starting point from which
    to reason whether the mathematical transformation is valid or not. Fig.~\ref{fig:Probability for two random variables} provides ideas on how to derivate the axioms.
    The finite random variable $X$ takes the value $x_{i}$ where $i$ can take values from $1,...,M$.
    The number of points inside a column define the frequency of an event $x_{i}$.
    The finite random variable $Y$ takes the value $y_{j}$ where $j$ can take values from $1,...,M$. The number of points inside a row defines the frequency of an event $y_{j}$.
    The probability of finite random variable $X = x_{i}$ independent of finite random variable Y is the number of points in column $c_{i}$ divided by the total number of points $N$.
    The probability of finite random variable $Y = y_{j}$ independent of finite random variable X, is the number of points in row $r_{j}$ divided by the total number of points $N$.
    The probability of finite random variable $X = x_{i}$ and $Y = y_{j}$ appearing at the same time is the number of points in cell $n_{ij}$ divided by the total number of $N$.
    The probability of finite random variable $X = x_{i}$ given $Y = y_{j}$ is the number of points in cell $n_{ij}$ divided by the number of points in row $r_{j}$. The probability
    of finite random variable $Y = y_{j}$ given $X = x_{i}$ is the number of points in cell $n_{ij}$ divided by the number of points in column $c_{i}$.
    The described axioms \cite[p.13 - 15]{bishop2006pattern} are shown in the following equations:
    \begin{IEEEeqnarray*}{rCl}
        P(S = s_{i}) = \frac{c_{i}}{N} \IEEEyesnumber \label{eq:1}  \\ \\
        P(W=w_{j}) = \frac{r_{j}}{N} \IEEEyesnumber \label{eq:2}  \\ \\
        P(W = w_{j}|S = s_{i}) = \frac{n_{ij}}{c_{i}} \IEEEyesnumber \label{eq:3} \\ \\
        P(S = s_{i}|W = w_{j}) = \frac{n_{ij}}{r_{j}} \IEEEyesnumber \label{eq:4} \\ \\
        P(S = s_{i}, W = w_{j}) = \frac{n_{ij}}{N} \IEEEyesnumber \label{eq:5} \\ \\
        P(S,W) = P(W,S) \IEEEyesnumber \label{eq:6}
    \end{IEEEeqnarray*}
    Based on the above, the following two equations can be derived:
    \begin{IEEEeqnarray*}{rCl}
        P(S = s_{i}|W^{C} = w^{c}_{i}) = \frac{(1 -\frac{n_{ij}}{c_{i}})\frac{c_{i}}{N}}{1 - \frac{r_{j}}{N}} \text{ see Eq.}\eqref{eq:-2} \IEEEyesnumber \label{eq:7} \\ \\
        P(W^{C} = w^{c}_{i}) = 1 - P(W) \IEEEyesnumber \label{eq:8}
    \end{IEEEeqnarray*}
    For switching between the two domains of probability and set theory the author uses the following definition\cite[p.~456]{rosen2012discrete}:
    \begin{IEEEeqnarray*}{rCl}
        P(S = s_{i}|W = w_{i})=\frac{p(S \cap W)}{P(W)} \IEEEyesnumber \label{eq:9}
    \end{IEEEeqnarray*}
    From this definition, the following two relations can be derived:
    \begin{IEEEeqnarray*}{rCl}
        P(S \cap W)=\frac{n_{ij}}{N}\text{ see Eq.~}\eqref{eq:-1} \IEEEyesnumber \label{eq:10} \\ \\
        \frac{P (S \cap W^{C})}{P(W^{C})} = \frac{\frac{c_{i}}{N} - \frac{n_{ij}}{N}}{1 - \frac{r_{j}}{N}}\text{ see Eq.~}\eqref{eq:-3} \IEEEyesnumber \label{eq:11}
    \end{IEEEeqnarray*}
    In order to map the input to these axioms, the equation checker uses an extended version of the parse\char`_expr function (explained in \hyperref[latexparser]{Subsection Latex parser}).
    To search and replace the axioms the author uses the data structure dictionary in which the regular expressions are defined as keys and the corresponding axioms as values.
    With the help of these regular expressions the parser searches the axioms in an input string and replaces them with the corresponding value. An application
    of the extended Latex parser can be seen in Fig.~\ref{fig:Application of dedicated extended Sympy parser}.
    Due to this substitution, Sympy is able to simplify the equations (\eqref{eq:12} - \eqref{eq:16}) to $\frac{n_{ij}}{c_{i}}$. At this point, if the mathematical transformation is correct, the equation check (see \ref{equationcheck}) returns 0.
    The overall process can be seen in Fig.~\ref{fig:Probability for two random variables using discrete probabilitites}.

    \begin{IEEEeqnarray*}{rCl}
        P(W|S)
        &=&\frac{P(S|W)*P(W)}{P(S)} \IEEEyesnumber \label{eq:12} \\
        & = & \frac{\frac{n_{ij}}{r_{j}} \cdot \frac{r_{j}}{N}}{\frac{c_{i}}{N}} \\
        & = & \frac{n_{ij}}{c_{i}} \\ \\
        P(W|S)
        &=&\frac{P(S|W)*P(W)}{P(S|W)*P(W)+P(S|W^{C})*P(W^{C})} \IEEEyesnumber \label{eq:13} \\
        & = & \frac{\frac{n_{ij}}{r_{j}} \cdot \frac{r_{j}}{N}}{\frac{n_{ij}}{r_{j}} \cdot \frac{r_{j}}{N}+\frac{(1 - \frac{n_{ij}}{c_{i}})*\frac{c_{i}}{N}}{1 - \frac{r_{j}}{N}}*(1 - \frac{r_{j}}{N})}\\
        & = & \frac{n_{ij}}{c_{i}} \\ \\
        P(W|S)
        &=&\frac{\frac{P(S \cap W)}{P(W)}*P(W)}{\frac{P(S \cap W)}{P(W)}*P(W) + \frac{P(S \cap W^{C})}{P(W^{C})}*P(W^{C})} \IEEEyesnumber \label{eq:14} \\
        & = & \frac{\frac{\frac{n_{ij}}{N}}{\frac{r_{j}}{N}}*\frac{r_{j}}{N}}{\frac{\frac{n_{ij}}{N}}{\frac{r_{j}}{N}}*\frac{r_{j}}{N} + \frac{\frac{c_{i}}{N} - \frac{n_{ij}}{N}}{1 - \frac{r_{j}}{N}}*(1 - \frac{r_{j}}{N})} \\
        & = & \frac{n_{ij}}{c_{i}} \\ \\
        P(W|S)
        &=&\frac{\frac{P(S \cap W)}{P(W)}*P(W)}{\frac{P(S \cap W)}{P(W)}*P(W) + \frac{P(S \cap W^{C})}{P(W^{C})}*P(W^{C})} \IEEEyesnumber \label{eq:15} \\
        & = & \frac{\frac{\frac{n_{ij}}{N}}{\frac{r_{j}}{N}}*\frac{r_{j}}{N}}{\frac{\frac{n_{ij}}{N}}{\frac{r_{j}}{N}}*\frac{r_{j}}{N} + \frac{\frac{c_{i}}{N} - \frac{n_{ij}}{N}}{1 - \frac{r_{j}}{N}}*(1 - \frac{r_{j}}{N})}\\
        & = & \frac{n_{ij}}{c_{i}} \\ \\
        P(W|S)
        &=&\frac{P(S \cap W)}{P(S)} \IEEEyesnumber \label{eq:16} \\
        & = & \frac{\frac{n}{N}}{\frac{n}{N}+\frac{c}{N} - \frac{n}{N}}\\
        & = &\frac{n_{ij}}{c_{i}}
    \end{IEEEeqnarray*}

    \begin{figure}[H]
        
    \begin{center}
        \begin{tikzpicture}[>=latex,line join=bevel,]

            \matrix (A) [matrix of nodes, nodes={draw, minimum size=18mm}, column sep=-\pgflinewidth]{
            $P(S,W):\frac{n}{r}$ & $P(W):\frac{r}{N}$ & $P(S):\frac{c}{N}$\\};
            \begin{scope}[node distance=3mm and 11mm]
                \draw [{Stealth}- ] (A-1-1.north)--++(90:5mm) node (PSW) [above] {$P(S,W)$};
                \path [{Stealth}- ] (A-1-1.north east)--++(90:6mm) node (PA) [above] {$\ast$};
                \draw [{Stealth}- ] (A-1-2.north)--++(90:5mm) node (PW) [above] {$P(W)$};
                \path [{Stealth}- ] (A-1-2.north east)--++(90:5mm) node (DV) [above] {$/$};
                \draw [{Stealth}- ] (A-1-3.north)--++(90:5mm) node (PS) [above] {$P(S)$};
                \draw [-{Stealth} ] (A-1-1.south)--++(-90:5mm) node (PSS) [below] {$\frac{n}{r}$};
                \path [-{Stealth} ] (A-1-1.south east)--++(-90:6mm) node (PAST1) [below] {$\ast$};
                \draw [-{Stealth} ] (A-1-2.south)--++(-90:5mm) node (PWST) [below] {$\frac{r}{N}$};
                \path [-{Stealth} ] (A-1-2.south east)--++(-90:6mm) node (PAST2) [below] {$\ast$};
                \draw [-{Stealth} ] (A-1-3.south)--++(-90:5mm) node (PWSTT) [below] {$\frac{c}{N}$};
            \end{scope}
            \begin{scope}[node distance=25mm and 1mm]
                \node[above left= of PSW] (frac) {$\char`\\ frac\{$};
                \node[above= of PSW] (LPSW) {$P(S|W)$};
                \node[above = of PW] (PWF) {$P(W)$};
                \node[above right= of PW] (SB) {$\}/\{$};
                \node[above = of PS] (PSL) {$P(S)$};
                \node[above right = of PS] (CB) {$\}$};
            \end{scope}
            \begin{scope}[node distance=26mm and 1mm]
                \node[above right= of PSW] (asterix1) {$*$};

            \end{scope}
            \draw [-{Stealth} ] (LPSW) -- (PSW);
            \draw [-{Stealth} ] (PWF) -- (PW);
            \draw [-{Stealth} ] (PSL) -- (PS);
            \node[left= of A] (HM) {Hashmap};
            \node[left= of frac] (IE) {Inserted equation};
            \node[left= of PSW] (PE) {Parsed equation};
            \node[above= of PE] (T) {Transformation};
            \node[below= of HM] (SI) {Sympy input};
            \node[below= of SI] (pe) {Parse expression};

            \begin{scope}[node distance=25mm and 11mm]
                \node[below= of PWST] (ghostnode) {};
            \end{scope}
            \draw [-{Stealth} ] (PWST) -- (ghostnode);
            \draw [-{Stealth} ] (PWSTT) -- (ghostnode);
            \draw [-{Stealth} ] (PSS) -- (ghostnode);
        \end{tikzpicture}
        \end{center}
        \hspace{200pt}
        \resizebox{!}{78mm}{
        \begin{tikzpicture}[>=latex,line join=bevel,]
        %%
          \node (Mul{Pow{Symbol{N}+ NegativeOne{}}+ Symbol{n}+ Pow{Symbol{r}+ NegativeOne{}}+ Symbol{r}+ Pow{Mul{Pow{Symbol{N}+ NegativeOne{}}+ Symbol{c}}+ NegativeOne{}}}_{}) at (98.0bp,231.0bp) [draw=black,ellipse] {Mul};
          \node (Symbol{n}_{0+}) at (9.0bp,175.5bp) [draw=black,ellipse] {n};
          \node (Pow{Symbol{r}+ NegativeOne{}}_{1+}) at (54.0bp,175.5bp) [draw=black,ellipse] {Pow};
          \node (Symbol{r}_{2+}) at (98.0bp,175.5bp) [draw=black,ellipse] {r};
          \node (Pow{Symbol{N}+ NegativeOne{}}_{3+}) at (142.0bp,175.5bp) [draw=black,ellipse] {Pow};
          \node (Pow{Mul{Pow{Symbol{N}+ NegativeOne{}}+ Symbol{c}}+ NegativeOne{}}_{4+}) at (196.0bp,175.5bp) [draw=black,ellipse] {Pow};
          \node (Symbol{r}_{1+ 0}) at (20.0bp,120.0bp) [draw=black,ellipse] {r};
          \node (NegativeOne{}_{1+ 1}) at (57.0bp,120.0bp) [draw=black,ellipse] {-1};
          \node (Symbol{N}_{3+ 0}) at (105.0bp,120.0bp) [draw=black,ellipse] {N};
          \node (NegativeOne{}_{3+ 1}) at (144.0bp,120.0bp) [draw=black,ellipse] {-1};
          \node (Mul{Pow{Symbol{N}+ NegativeOne{}}+ Symbol{c}}_{4+ 0}) at (194.0bp,120.0bp) [draw=black,ellipse] {Mul};
          \node (NegativeOne{}_{4+ 1}) at (240.0bp,120.0bp) [draw=black,ellipse] {-1};
          \node (Symbol{c}_{4+ 0+ 0}) at (172.0bp,64.5bp) [draw=black,ellipse] {c};
          \node (Pow{Symbol{N}+ NegativeOne{}}_{4+ 0+ 1}) at (216.0bp,64.5bp) [draw=black,ellipse] {Pow};
          \node (Symbol{N}_{4+ 0+ 1+ 0}) at (196.0bp,9.5bp) [draw=black,ellipse] {N};
          \node (NegativeOne{}_{4+ 0+ 1+ 1}) at (235.0bp,9.5bp) [draw=black,ellipse] {-1};
          \draw [->] (Mul{Pow{Symbol{N}+ NegativeOne{}}+ Symbol{n}+ Pow{Symbol{r}+ NegativeOne{}}+ Symbol{r}+ Pow{Mul{Pow{Symbol{N}+ NegativeOne{}}+ Symbol{c}}+ NegativeOne{}}}_{}) ..controls (70.304bp,213.73bp) and (42.754bp,196.55bp)  .. (Symbol{n}_{0+});
          \draw [->] (Mul{Pow{Symbol{N}+ NegativeOne{}}+ Symbol{n}+ Pow{Symbol{r}+ NegativeOne{}}+ Symbol{r}+ Pow{Mul{Pow{Symbol{N}+ NegativeOne{}}+ Symbol{c}}+ NegativeOne{}}}_{}) ..controls (84.455bp,213.92bp) and (75.212bp,202.26bp)  .. (Pow{Symbol{r}+ NegativeOne{}}_{1+});
          \draw [->] (Mul{Pow{Symbol{N}+ NegativeOne{}}+ Symbol{n}+ Pow{Symbol{r}+ NegativeOne{}}+ Symbol{r}+ Pow{Mul{Pow{Symbol{N}+ NegativeOne{}}+ Symbol{c}}+ NegativeOne{}}}_{}) ..controls (98.0bp,213.34bp) and (98.0bp,202.96bp)  .. (Symbol{r}_{2+});
          \draw [->] (Mul{Pow{Symbol{N}+ NegativeOne{}}+ Symbol{n}+ Pow{Symbol{r}+ NegativeOne{}}+ Symbol{r}+ Pow{Mul{Pow{Symbol{N}+ NegativeOne{}}+ Symbol{c}}+ NegativeOne{}}}_{}) ..controls (111.54bp,213.92bp) and (120.79bp,202.26bp)  .. (Pow{Symbol{N}+ NegativeOne{}}_{3+});
          \draw [->] (Mul{Pow{Symbol{N}+ NegativeOne{}}+ Symbol{n}+ Pow{Symbol{r}+ NegativeOne{}}+ Symbol{r}+ Pow{Mul{Pow{Symbol{N}+ NegativeOne{}}+ Symbol{c}}+ NegativeOne{}}}_{}) ..controls (126.61bp,214.8bp) and (155.21bp,198.6bp)  .. (Pow{Mul{Pow{Symbol{N}+ NegativeOne{}}+ Symbol{c}}+ NegativeOne{}}_{4+});
          \draw [->] (Pow{Symbol{r}+ NegativeOne{}}_{1+}) ..controls (43.293bp,158.02bp) and (35.678bp,145.59bp)  .. (Symbol{r}_{1+ 0});
          \draw [->] (Pow{Symbol{r}+ NegativeOne{}}_{1+}) ..controls (54.91bp,158.67bp) and (55.451bp,148.66bp)  .. (NegativeOne{}_{1+ 1});
          \draw [->] (Pow{Symbol{N}+ NegativeOne{}}_{3+}) ..controls (130.54bp,158.31bp) and (122.63bp,146.44bp)  .. (Symbol{N}_{3+ 0});
          \draw [->] (Pow{Symbol{N}+ NegativeOne{}}_{3+}) ..controls (142.61bp,158.67bp) and (142.97bp,148.66bp)  .. (NegativeOne{}_{3+ 1});
          \draw [->] (Pow{Mul{Pow{Symbol{N}+ NegativeOne{}}+ Symbol{c}}+ NegativeOne{}}_{4+}) ..controls (195.4bp,158.84bp) and (195.05bp,149.15bp)  .. (Mul{Pow{Symbol{N}+ NegativeOne{}}+ Symbol{c}}_{4+ 0});
          \draw [->] (Pow{Mul{Pow{Symbol{N}+ NegativeOne{}}+ Symbol{c}}+ NegativeOne{}}_{4+}) ..controls (209.61bp,158.34bp) and (219.48bp,145.89bp)  .. (NegativeOne{}_{4+ 1});
          \draw [->] (Mul{Pow{Symbol{N}+ NegativeOne{}}+ Symbol{c}}_{4+ 0}) ..controls (186.94bp,102.19bp) and (182.5bp,90.982bp)  .. (Symbol{c}_{4+ 0+ 0});
          \draw [->] (Mul{Pow{Symbol{N}+ NegativeOne{}}+ Symbol{c}}_{4+ 0}) ..controls (200.88bp,102.65bp) and (205.0bp,92.26bp)  .. (Pow{Symbol{N}+ NegativeOne{}}_{4+ 0+ 1});
          \draw [->] (Pow{Symbol{N}+ NegativeOne{}}_{4+ 0+ 1}) ..controls (209.85bp,47.576bp) and (206.08bp,37.211bp)  .. (Symbol{N}_{4+ 0+ 1+ 0});
          \draw [->] (Pow{Symbol{N}+ NegativeOne{}}_{4+ 0+ 1}) ..controls (221.85bp,47.576bp) and (225.43bp,37.211bp)  .. (NegativeOne{}_{4+ 0+ 1+ 1});
        %
        \end{tikzpicture}
        }
        \caption{Transformation of random variables into the Sympy expression tree}
        \caption*{Here the transformation from input into discrete probabilities can be seen. This figure is nearly identical to Fig.~\ref{fig:Sympy expression tree with specific values}
        except for the discrete probabilities.}
        \label{fig:Probability for two random variables using discrete probabilitites}
    \end{figure}
\end{document}
