%!TEX root = ../report.tex

\begin{document}
\chapter{Introduction}
    The number of students at university is steadily increasing, therefore grading assignments, and exams consumes more and more time.
    One approach in computer science to tackle this problem is to use computer-assisted grading. Students code their assignments and exams in e.g., Jupyter notebook.
    Writing assignments and exams in a digital form allows the examiner to grade them more quickly.\\ Mathematics is part of most study programs and a challeging field to automate
    in the context of computer-assisted grading. At present, computer-assisted grading for mathematics usually involves validating a single result or validating multiple choice answers.
    However when talking about automatic single value validation, computer-assisted grading systems nowadays only check
    the final result, they do not check intermediate steps which take place before the final result. This becomes relevant when grading
    a step-by-step proof, or a transformation of equations. Often the student does not submit the right solution, but nevertheless parts of their calculations are correct
    - some points should be awarded for these correct steps. \newpage Here the following question arises: What can be defined as a proof in the context of this project? An example of step-by-step
    proof from the domain of probability theory shows the following equations. The proof uses a set of known transformations (e.g., Bayes theorem is given in Eq.~\eqref{eq:0}) to transform one side into the other.
    \begin{IEEEeqnarray*}{rCl}
    P(W|S)
    &=& \frac{P(S|W) P(W)}{P(S)} \\ \\
    & = & \frac{P(S|W)P(W)}{P(S|W)P(W)+P(S|W^{c})P(W^{c})} \\ \\
    & = & \frac{\frac{P(S \cap W)}{P(W)}P(W)}{\frac{P(S \cap W)}{P(W)}P(W) + \frac{P(S \cap W^{c})}{P(W^{c})}P(W^{c})} \IEEEyesnumber \label{eq:0} \\ \\
    & = & \frac{P(S \cap W)}{P(S \cap W)+P(S \cap W^{c})} \\ \\
    & = & \frac{P(S \cap W)}{P(S)} \\ \\
    & = & P(W|S)
\end{IEEEeqnarray*}
\end{document}