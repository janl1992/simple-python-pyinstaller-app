%!TEX root = ../report.tex
\begin{document}
    \chapter{Equation checker for contingency table}
    Until now every approach discussed in the previous chapters has focused on one specific class of mathematical proof. This limits the number of different assignments which can be graded automatically .
    In order to provide a variety of assignments, the author uses a contingency table where one random variable is independent of the other random variable (see Fig.~\ref{fig:FourFieldGraphic}). This constraint
    can be formalized as follows:
    \begin{IEEEeqnarray*}{rCl}
        P(A \cap B)&=&P(A)P(B) \label{eq:25}
    \end{IEEEeqnarray*}
    Due to this it is possible to show that, based on one field the random variables in all adjacent fields are also independent. The following two
    examples demonstrate this:
    \begin{IEEEeqnarray*}{rCl}
        $\text{Given: }$P(A \cap B)&=&P(A)P(B)$$\\
        P(A)&=&P(A \cap B) + P(A \cap B^{C}) \\
        \iff P(A) - P(A \cap B)&=&P(A \cap B^{C}) \IEEEyesnumber \label{eq:26} \\
        \iff P(A) - P(A)P(B)&=&P(A \cap B^{C}) \\
        \iff P(A)(1 - P(B))&=&P(A \cap B^{C}) \\
        \iff P(A)P(B^{C})&=&P(A \cap B^{C})
    \end{IEEEeqnarray*}
    \begin{IEEEeqnarray*}{rCl}
        $\text{Given: }$P(A^{C} \cap B)&=&P(A^{C})P(B)$$\\
        P(A^{C})&=&P(A^{C} \cap B) + P(A^{C} \cap B^{C}) \\
        \iff P(A^{C}) - P(A^{C} \cap B)&=&P(A^{C} \cap B^{C}) \IEEEyesnumber \label{eq:27} \\
        \iff P(A^{C}) - P(A^{C})P(B)&=&P(A^{C} \cap B^{C}) \\
        \iff P(A^{C})(1 - P(B))&=&P(A^{C} \cap B^{C}) \\
        \iff P(A^{C})P(B^{C})&=&P(A^{C} \cap B^{C})
    \end{IEEEeqnarray*}
    As can be seen in these two equations firstly comparing the initial equation with the equation of interest (see \ref{fig:CheckArray}), then simplifying them and checking whether the difference
    between them is equal to zero, does not produce the desired result here. This is due to the equivalence transformations, which are used in Eq.~\eqref{eq:26} and \eqref{eq:27}. Drawing a comparison with
    the initial equation would only return a valid check during the first check (in Eq.~\eqref{eq:27} this would be $P(A^{C}) = P(A^{C} \cap B) + P(A^{C} \cap B^{C})$). After
    the first equivalence transformation the equation of interest is no longer equal to the initial equation (compare in Eq.~\eqref{eq:27} the third line $P(A^{C} \cap B^{C}) \neq P(A^{C})$).
    Therefore the equation check was carried out line by line. The equation checker validates whether two equations are equal for each line. There is no overall check
    for equality between two line and could be a topic for further research. In order to implement this, the equation checker extracts the random variables, generates
    discrete probabilities and then compares this equation to the equation in the same line using simplify. The overall algorithm can be seen in Fig.~\ref{fig:FourFieldGraphic}.

    \begin{figure}
        \resizebox{!}{101mm}{
\begin{tikzpicture}[every node/.style={anchor=base,text depth=0.5ex,text height=1ex,text width=5em}, text centered]
    \matrix (A) [matrix of nodes, nodes={draw, minimum size=18mm}, column sep=-\pgflinewidth, ampersand replacement=\&]{
    $\scriptscriptstyle P(A)*P(B)$ \& $\times$ \\ $\times$ \& $\times$\\};
    %         \draw [{Stealth}-] (A-1-1.east)--++(-180:5mm) (A-1-2);
    \begin{scope}[node distance=.5mm and .5mm]
        \node [left=of A-1-1] (NA) {$A$};
        \node [left=of A-2-1] (NCA) {$\bar{A}$};
        \node [above=of A-1-1] (NB) {$B$};
        \node [above=of A-1-2] (NCB) {$\bar{B}$};
        \node [right=of A-1-2] (NPA) {$P(A)$};
        \node [right=of A-2-2] (NPBA) {$P(\bar{A})$};
    \end{scope}
    \begin{scope}[node distance=3mm and 3mm]
        \node [below=of A-2-1] (NPB) {$P(B)$};
        \node [below=of A-2-2] (NCBB) {$P(\bar{B})$};
    \end{scope}
    \begin{scope}[node distance=.8mm and .8mm]
        \node [above right=of A-1-2] (NPA) {$\sum$};
    \end{scope}
    \path [-{Stealth} ] (A-2-1.south east)--++(-90:10mm) node (NamenodeA) [below] {Table $A$};
    \begin{scope}[node distance=40mm and 40mm]
        \matrix (B) [matrix of nodes, nodes={draw, minimum size=18mm}, column sep=-\pgflinewidth, right=of A, ampersand replacement=\&]{
        $\times$ \& $\scriptscriptstyle P(A)*P(B^{C})$ \\ $\times$ \& $\times$ \\};
    \end{scope}
    \begin{scope}[node distance=.5mm and .5mm]
        \node [left=of B-1-1] (NA) {$A$};
        \node [left=of B-2-1] (NCA) {$\bar{A}$};
        \node [above=of B-1-1] (NB) {$B$};
        \node [above=of B-1-2] (NCB) {$\bar{B}$};
        \node [right=of B-1-2] (NPA) {$P(A)$};
        \node [right=of B-2-2] (NPBA) {$P(\bar{A})$};
    \end{scope}
    \begin{scope}[node distance=3mm and 3mm]
        \node [below=of B-2-1] (NPB) {$P(B)$};
        \node [below=of B-2-2] (NCBB) {$P(\bar{B})$};
    \end{scope}
    \path [-{Stealth} ] (B-2-1.south east)--++(-90:10mm) node (NamenodeB) [below] {Table $B$};

    \begin{scope}[node distance=.8mm and .8mm]
        \node [above right=of B-1-2] (NPA) {$\sum$};
    \end{scope}
    \begin{scope}[node distance=30mm and 30mm]
        \matrix (C) [matrix of nodes, nodes={draw, minimum size=18mm}, column sep=-\pgflinewidth, below=of A, ampersand replacement=\&]{
        $\times$ \& $\times$ \\ $\scriptscriptstyle P(A^{C})*P(B)$ \& $\times$ \\};
    \end{scope}
    \begin{scope}[node distance=.5mm and .5mm]
        \node [left=of C-1-1] (NA) {$A$};
        \node [above=of C-1-2] (NCB) {$\bar{B}$};
        \node [left=of C-2-1] (NCA) {$\bar{A}$};
        \node [above=of C-1-1] (NB) {$B$};
        \node [right=of C-1-2] (NPA) {$P(A)$};
        \node [right=of C-2-2] (NPBA) {$P(\bar{A})$};
    \end{scope}
    \begin{scope}[node distance=3mm and 3mm]
        \node [below=of C-2-1] (NPB) {$P(B)$};
        \node [below=of C-2-2] (NCBB) {$P(\bar{B})$};
    \end{scope}
    \begin{scope}[node distance=.8mm and .8mm]
        \node [above right=of C-1-2] (NPA) {$\sum$};
    \end{scope}
    \path [-{Stealth} ] (C-2-1.south east)--++(-90:10mm) node (NamenodeC) [below] {Table $C$};
    \begin{scope}[node distance=30mm and 30mm]
        \matrix (D) [matrix of nodes, nodes={draw, minimum size=18mm}, column sep=-\pgflinewidth, below=of B, ampersand replacement=\&]{
        $\times$ \& $\times$ \\ $\times$ \& $\scriptscriptstyle P(A^{C})*P(B^{C})$ \\};
    \end{scope}
    \begin{scope}[node distance=.5mm and .5mm]
        \node [left=of D-1-1] (NA) {$A$};
        \node [above=of D-1-2] (NCB) {$\bar{B}$};
        \node [left=of D-2-1] (NCA) {$\bar{A}$};
        \node [above=of D-1-1] (NB) {$B$};
        \node [right=of D-1-2] (NPA) {$P(A)$};
        \node [right=of D-2-2] (NPBA) {$P(\bar{A})$};
    \end{scope}
    \begin{scope}[node distance=3mm and 3mm]
        \node [below=of D-2-1] (NPB) {$P(B)$};
        \node [below=of D-2-2] (NCBB) {$P(\bar{B})$};
    \end{scope}
    \begin{scope}[node distance=.8mm and .8mm]
        \node [above right=of D-1-2] (NPA) {$\sum$};
    \end{scope}
    \path [-{Stealth} ] (D-2-1.south east)--++(-90:10mm) node (NamenodeD) [below] {Table $D$};

\end{tikzpicture}
}
        \caption{Contingency tables}
        \caption*{This figure shows a contigency table with the constraint $P(A \cap B) = P(A)P(B)$.
        One field demonstrates the probability that two independent events happen at the same time.
        Due to the constraint that the random variables are independent of each other, it is possible to show that the random variables in
        all adjacent fields are also independent. $\times$ illustrates all the possibilities to reach the conclusion that the random variables in the adjacent fields are independent.
        This can be carried out 12 times (there are 4 fields, in each of which 3 derivations are possible). In this way
        a variety of assignments can be created.}
        \label{fig:FourFieldGraphic}
    \end{figure}

    \begin{figure}
        
    \begin{center}
        \begin{tikzpicture}[>=latex,line join=bevel,]

            \matrix (A) [matrix of nodes, nodes={draw, minimum size=18mm, anchor=center}, column sep=-\pgflinewidth]{
            $P(A \cap B):\frac{r_{1}}{N}*\frac{r_{2}}{N}$ & $P(A)*P(B):\frac{r_{1}}{N}*\frac{r_{2}}{N}$ & $P(B^{C} \cap A^{C}):\frac{1-r_{1}}{N}*\frac{1-r_{2}}{N}$\\};
            \begin{scope}[node distance=3mm and 11mm]
                \draw [{Stealth}- ] (A-1-1.north)--++(90:5mm) node (PSW) [above] {$P(A \cap B)$};
                %                \path [{Stealth}- ] (A-1-1.north east)--++(90:6mm) node (PA) [above] {$\ast$};
                %                \draw [{Stealth}- ] (A-1-2.north)--++(90:6mm) node (PW) [above] {$+$};
                %                \path [{Stealth}- ] (A-1-2.north east)--++(90:5mm) node (DV) [above] {$/$};
                \draw [{Stealth}- ] (A-1-3.north)--++(90:5mm) node (PS) [above] {$P(B^{C} \cap A^{C})$};
                \path [-{Stealth} ] (A-1-3.north east)--++(90:5mm) node (PSTT) [above] {$$};
                \draw [-{Stealth} ] (A-1-1.south)--++(-90:5mm) node (PSS) [below] {$\frac{r_{1}}{N}*\frac{r_{2}}{N}$};
                %                \path [-{Stealth} ] (A-1-1.south east)--++(-90:6mm) node (PAST1) [below] {$\ast$};
                \path [-{Stealth} ] (A-1-2.south)--++(-90:5mm) node (PWST) [below] {$+$};
                \path [-{Stealth} ] (A-1-2.north)--++(90:5mm) node (PWSTA) [above] {$+$};
                %                \path [-{Stealth} ] (A-1-2.south east)--++(-90:6mm) node (PAST2) [below] {$\ast$};
                \draw [-{Stealth} ] (A-1-3.south)--++(-90:5mm) node (PWSTT) [below] {$\frac{1-r_{1}}{N}*\frac{1-r_{2}}{N}$};

            \end{scope}
%            \begin{scope}[node distance=25.5mm and 1mm]
%                \node[above= of PSW] (LPSW) {$A \cap B$};
%                \node[above= of PS] (PWF) {$B^{C} \cap A^{C}$};
%            \end{scope}
%            \begin{scope}[node distance=25mm and 1mm]
%                \node[above left= of PSW] (frac) {$P($};
%                \node[above= of PWSTA] (asterix1) {$)+P($};
%            \end{scope}
%            \begin{scope}[node distance=28mm and 1mm]
%                \node[above right= of PSTT] (PWFR) {$)$};
%            \end{scope}
%            \draw [-{Stealth} ] (LPSW) -- (PSW);
%            \draw [-{Stealth} ] (PWF) -- (PS);
            \node[left= of A] (HM) {Hashmap};
%            \node[left= of frac] (IE) {Inserted equation};
            %            \node[below= of A] (PE) {Transformation};
            \node[left= of PSW] (PE) {Parsed equation};
%            \node[above= of PE] (T) {Transformation};
            %            \node[above= of T] (IE) {Initial equation};
            \node[below= of HM] (SI) {Sympy input};
            \node[below= of SI] (pe) {Parse expression};

            \begin{scope}[node distance=25mm and 11mm]
                \node[below= of PWST] (ghostnode) {};
            \end{scope}
                        \draw [-{Stealth} ] (PWST) -- (ghostnode);
                        \draw [-{Stealth} ] (PSS) -- (ghostnode);
                        \draw [-{Stealth} ] (PWSTT) -- (ghostnode);
        \end{tikzpicture}
    \end{center}
    \hspace{145pt}
    \resizebox{!}{88mm}{
    \begin{tikzpicture}[>=latex,line join=bevel,]
        %%
        \node (Add{Mul{Add{One{}+ Mul{NegativeOne{}+ Mul{Pow{Symbol{N}+ NegativeOne{}}+ Symbol{r0}}}}+ Add{One{}+ Mul{NegativeOne{}+ Mul{Pow{Symbol{N}+ NegativeOne{}}+ Symbol{r1}}}}}+ Mul{Pow{Symbol{N}+ NegativeOne{}}+ Pow{Symbol{N}+ NegativeOne{}}+ Symbol{r0}+ Symbol{r1}}}_{}) at (155.0bp,344.0bp) [draw=black,ellipse] {Add};
        \node (Mul{Pow{Symbol{N}+ NegativeOne{}}+ Pow{Symbol{N}+ NegativeOne{}}+ Symbol{r0}+ Symbol{r1}}_{0+}) at (105.0bp,288.0bp) [draw=black,ellipse] {Mul};
        \node (Mul{Add{One{}+ Mul{NegativeOne{}+ Mul{Pow{Symbol{N}+ NegativeOne{}}+ Symbol{r0}}}}+ Add{One{}+ Mul{NegativeOne{}+ Mul{Pow{Symbol{N}+ NegativeOne{}}+ Symbol{r1}}}}}_{1+}) at (215.0bp,288.0bp) [draw=black,ellipse] {Mul};
        \node (Symbol{r0}_{0+ 0}) at (11.0bp,232.0bp) [draw=black,ellipse] {r0};
        \node (Pow{Symbol{N}+ NegativeOne{}}_{0+ 1}) at (58.0bp,232.0bp) [draw=black,ellipse] {Pow};
        \node (Symbol{r1}_{0+ 2}) at (105.0bp,232.0bp) [draw=black,ellipse] {r1};
        \node (Pow{Symbol{N}+ NegativeOne{}}_{0+ 3}) at (152.0bp,232.0bp) [draw=black,ellipse] {Pow};
        \node (Symbol{N}_{0+ 1+ 0}) at (27.0bp,176.0bp) [draw=black,ellipse] {N};
        \node (NegativeOne{}_{0+ 1+ 1}) at (66.0bp,176.0bp) [draw=black,ellipse] {-1};
        \node (Symbol{N}_{0+ 3+ 0}) at (113.0bp,176.0bp) [draw=black,ellipse] {N};
        \node (NegativeOne{}_{0+ 3+ 1}) at (152.0bp,176.0bp) [draw=black,ellipse] {-1};
        \node (Add{One{}+ Mul{NegativeOne{}+ Mul{Pow{Symbol{N}+ NegativeOne{}}+ Symbol{r1}}}}_{1+ 0}) at (215.0bp,232.0bp) [draw=black,ellipse] {Add};
        \node (Add{One{}+ Mul{NegativeOne{}+ Mul{Pow{Symbol{N}+ NegativeOne{}}+ Symbol{r0}}}}_{1+ 1}) at (278.0bp,232.0bp) [draw=black,ellipse] {Add};
        \node (One{}_{1+ 0+ 0}) at (190.0bp,176.0bp) [draw=black,ellipse] {1};
        \node (Mul{NegativeOne{}+ Mul{Pow{Symbol{N}+ NegativeOne{}}+ Symbol{r1}}}_{1+ 0+ 1}) at (234.0bp,176.0bp) [draw=black,ellipse] {Mul};
        \node (NegativeOne{}_{1+ 0+ 1+ 0}) at (205.0bp,120.0bp) [draw=black,ellipse] {-1};
        \node (Mul{Pow{Symbol{N}+ NegativeOne{}}+ Symbol{r1}}_{1+ 0+ 1+ 1}) at (251.0bp,120.0bp) [draw=black,ellipse] {Mul};
        \node (Symbol{r1}_{1+ 0+ 1+ 1+ 0}) at (223.0bp,64.5bp) [draw=black,ellipse] {r1};
        \node (Pow{Symbol{N}+ NegativeOne{}}_{1+ 0+ 1+ 1+ 1}) at (270.0bp,64.5bp) [draw=black,ellipse] {Pow};
        \node (Symbol{N}_{1+ 0+ 1+ 1+ 1+ 0}) at (262.0bp,9.5bp) [draw=black,ellipse] {N};
        \node (NegativeOne{}_{1+ 0+ 1+ 1+ 1+ 1}) at (301.0bp,9.5bp) [draw=black,ellipse] {-1};
        \node (One{}_{1+ 1+ 0}) at (278.0bp,176.0bp) [draw=black,ellipse] {1};
        \node (Mul{NegativeOne{}+ Mul{Pow{Symbol{N}+ NegativeOne{}}+ Symbol{r0}}}_{1+ 1+ 1}) at (322.0bp,176.0bp) [draw=black,ellipse] {Mul};
        \node (NegativeOne{}_{1+ 1+ 1+ 0}) at (305.0bp,120.0bp) [draw=black,ellipse] {-1};
        \node (Mul{Pow{Symbol{N}+ NegativeOne{}}+ Symbol{r0}}_{1+ 1+ 1+ 1}) at (351.0bp,120.0bp) [draw=black,ellipse] {Mul};
        \node (Symbol{r0}_{1+ 1+ 1+ 1+ 0}) at (332.0bp,64.5bp) [draw=black,ellipse] {r0};
        \node (Pow{Symbol{N}+ NegativeOne{}}_{1+ 1+ 1+ 1+ 1}) at (379.0bp,64.5bp) [draw=black,ellipse] {Pow};
        \node (Symbol{N}_{1+ 1+ 1+ 1+ 1+ 0}) at (348.0bp,9.5bp) [draw=black,ellipse] {N};
        \node (NegativeOne{}_{1+ 1+ 1+ 1+ 1+ 1}) at (387.0bp,9.5bp) [draw=black,ellipse] {-1};
        \draw [->] (Add{Mul{Add{One{}+ Mul{NegativeOne{}+ Mul{Pow{Symbol{N}+ NegativeOne{}}+ Symbol{r0}}}}+ Add{One{}+ Mul{NegativeOne{}+ Mul{Pow{Symbol{N}+ NegativeOne{}}+ Symbol{r1}}}}}+ Mul{Pow{Symbol{N}+ NegativeOne{}}+ Pow{Symbol{N}+ NegativeOne{}}+ Symbol{r0}+ Symbol{r1}}}_{}) ..controls (139.72bp,326.89bp) and (128.87bp,314.74bp)  .. (Mul{Pow{Symbol{N}+ NegativeOne{}}+ Pow{Symbol{N}+ NegativeOne{}}+ Symbol{r0}+ Symbol{r1}}_{0+});
        \draw [->] (Add{Mul{Add{One{}+ Mul{NegativeOne{}+ Mul{Pow{Symbol{N}+ NegativeOne{}}+ Symbol{r0}}}}+ Add{One{}+ Mul{NegativeOne{}+ Mul{Pow{Symbol{N}+ NegativeOne{}}+ Symbol{r1}}}}}+ Mul{Pow{Symbol{N}+ NegativeOne{}}+ Pow{Symbol{N}+ NegativeOne{}}+ Symbol{r0}+ Symbol{r1}}}_{}) ..controls (173.5bp,326.73bp) and (187.55bp,313.62bp)  .. (Mul{Add{One{}+ Mul{NegativeOne{}+ Mul{Pow{Symbol{N}+ NegativeOne{}}+ Symbol{r0}}}}+ Add{One{}+ Mul{NegativeOne{}+ Mul{Pow{Symbol{N}+ NegativeOne{}}+ Symbol{r1}}}}}_{1+});
        \draw [->] (Mul{Pow{Symbol{N}+ NegativeOne{}}+ Pow{Symbol{N}+ NegativeOne{}}+ Symbol{r0}+ Symbol{r1}}_{0+}) ..controls (76.438bp,270.98bp) and (47.621bp,253.82bp)  .. (Symbol{r0}_{0+ 0});
        \draw [->] (Mul{Pow{Symbol{N}+ NegativeOne{}}+ Pow{Symbol{N}+ NegativeOne{}}+ Symbol{r0}+ Symbol{r1}}_{0+}) ..controls (90.556bp,270.79bp) and (80.197bp,258.45bp)  .. (Pow{Symbol{N}+ NegativeOne{}}_{0+ 1});
        \draw [->] (Mul{Pow{Symbol{N}+ NegativeOne{}}+ Pow{Symbol{N}+ NegativeOne{}}+ Symbol{r0}+ Symbol{r1}}_{0+}) ..controls (105.0bp,270.73bp) and (105.0bp,260.68bp)  .. (Symbol{r1}_{0+ 2});
        \draw [->] (Mul{Pow{Symbol{N}+ NegativeOne{}}+ Pow{Symbol{N}+ NegativeOne{}}+ Symbol{r0}+ Symbol{r1}}_{0+}) ..controls (119.44bp,270.79bp) and (129.8bp,258.45bp)  .. (Pow{Symbol{N}+ NegativeOne{}}_{0+ 3});
        \draw [->] (Pow{Symbol{N}+ NegativeOne{}}_{0+ 1}) ..controls (48.496bp,214.83bp) and (42.059bp,203.2bp)  .. (Symbol{N}_{0+ 1+ 0});
        \draw [->] (Pow{Symbol{N}+ NegativeOne{}}_{0+ 1}) ..controls (60.397bp,215.22bp) and (61.867bp,204.93bp)  .. (NegativeOne{}_{0+ 1+ 1});
        \draw [->] (Pow{Symbol{N}+ NegativeOne{}}_{0+ 3}) ..controls (140.0bp,214.76bp) and (131.37bp,202.37bp)  .. (Symbol{N}_{0+ 3+ 0});
        \draw [->] (Pow{Symbol{N}+ NegativeOne{}}_{0+ 3}) ..controls (152.0bp,214.94bp) and (152.0bp,204.69bp)  .. (NegativeOne{}_{0+ 3+ 1});
        \draw [->] (Mul{Add{One{}+ Mul{NegativeOne{}+ Mul{Pow{Symbol{N}+ NegativeOne{}}+ Symbol{r0}}}}+ Add{One{}+ Mul{NegativeOne{}+ Mul{Pow{Symbol{N}+ NegativeOne{}}+ Symbol{r1}}}}}_{1+}) ..controls (215.0bp,270.91bp) and (215.0bp,261.17bp)  .. (Add{One{}+ Mul{NegativeOne{}+ Mul{Pow{Symbol{N}+ NegativeOne{}}+ Symbol{r1}}}}_{1+ 0});
        \draw [->] (Mul{Add{One{}+ Mul{NegativeOne{}+ Mul{Pow{Symbol{N}+ NegativeOne{}}+ Symbol{r0}}}}+ Add{One{}+ Mul{NegativeOne{}+ Mul{Pow{Symbol{N}+ NegativeOne{}}+ Symbol{r1}}}}}_{1+}) ..controls (234.1bp,271.03bp) and (248.94bp,257.83bp)  .. (Add{One{}+ Mul{NegativeOne{}+ Mul{Pow{Symbol{N}+ NegativeOne{}}+ Symbol{r0}}}}_{1+ 1});
        \draw [->] (Add{One{}+ Mul{NegativeOne{}+ Mul{Pow{Symbol{N}+ NegativeOne{}}+ Symbol{r1}}}}_{1+ 0}) ..controls (207.15bp,214.42bp) and (202.16bp,203.23bp)  .. (One{}_{1+ 0+ 0});
        \draw [->] (Add{One{}+ Mul{NegativeOne{}+ Mul{Pow{Symbol{N}+ NegativeOne{}}+ Symbol{r1}}}}_{1+ 0}) ..controls (220.88bp,214.66bp) and (224.33bp,204.49bp)  .. (Mul{NegativeOne{}+ Mul{Pow{Symbol{N}+ NegativeOne{}}+ Symbol{r1}}}_{1+ 0+ 1});
        \draw [->] (Mul{NegativeOne{}+ Mul{Pow{Symbol{N}+ NegativeOne{}}+ Symbol{r1}}}_{1+ 0+ 1}) ..controls (224.9bp,158.43bp) and (219.12bp,147.26bp)  .. (NegativeOne{}_{1+ 0+ 1+ 0});
        \draw [->] (Mul{NegativeOne{}+ Mul{Pow{Symbol{N}+ NegativeOne{}}+ Symbol{r1}}}_{1+ 0+ 1}) ..controls (239.26bp,158.66bp) and (242.35bp,148.49bp)  .. (Mul{Pow{Symbol{N}+ NegativeOne{}}+ Symbol{r1}}_{1+ 0+ 1+ 1});
        \draw [->] (Mul{Pow{Symbol{N}+ NegativeOne{}}+ Symbol{r1}}_{1+ 0+ 1+ 1}) ..controls (242.26bp,102.67bp) and (236.75bp,91.747bp)  .. (Symbol{r1}_{1+ 0+ 1+ 1+ 0});
        \draw [->] (Mul{Pow{Symbol{N}+ NegativeOne{}}+ Symbol{r1}}_{1+ 0+ 1+ 1}) ..controls (256.91bp,102.73bp) and (260.42bp,92.483bp)  .. (Pow{Symbol{N}+ NegativeOne{}}_{1+ 0+ 1+ 1+ 1});
        \draw [->] (Pow{Symbol{N}+ NegativeOne{}}_{1+ 0+ 1+ 1+ 1}) ..controls (267.57bp,47.822bp) and (266.13bp,37.903bp)  .. (Symbol{N}_{1+ 0+ 1+ 1+ 1+ 0});
        \draw [->] (Pow{Symbol{N}+ NegativeOne{}}_{1+ 0+ 1+ 1+ 1}) ..controls (279.73bp,47.245bp) and (285.91bp,36.271bp)  .. (NegativeOne{}_{1+ 0+ 1+ 1+ 1+ 1});
        \draw [->] (Add{One{}+ Mul{NegativeOne{}+ Mul{Pow{Symbol{N}+ NegativeOne{}}+ Symbol{r0}}}}_{1+ 1}) ..controls (278.0bp,214.73bp) and (278.0bp,204.68bp)  .. (One{}_{1+ 1+ 0});
        \draw [->] (Add{One{}+ Mul{NegativeOne{}+ Mul{Pow{Symbol{N}+ NegativeOne{}}+ Symbol{r0}}}}_{1+ 1}) ..controls (291.53bp,214.78bp) and (300.75bp,203.04bp)  .. (Mul{NegativeOne{}+ Mul{Pow{Symbol{N}+ NegativeOne{}}+ Symbol{r0}}}_{1+ 1+ 1});
        \draw [->] (Mul{NegativeOne{}+ Mul{Pow{Symbol{N}+ NegativeOne{}}+ Symbol{r0}}}_{1+ 1+ 1}) ..controls (316.71bp,158.57bp) and (313.57bp,148.23bp)  .. (NegativeOne{}_{1+ 1+ 1+ 0});
        \draw [->] (Mul{NegativeOne{}+ Mul{Pow{Symbol{N}+ NegativeOne{}}+ Symbol{r0}}}_{1+ 1+ 1}) ..controls (331.0bp,158.62bp) and (336.61bp,147.79bp)  .. (Mul{Pow{Symbol{N}+ NegativeOne{}}+ Symbol{r0}}_{1+ 1+ 1+ 1});
        \draw [->] (Mul{Pow{Symbol{N}+ NegativeOne{}}+ Symbol{r0}}_{1+ 1+ 1+ 1}) ..controls (345.09bp,102.73bp) and (341.58bp,92.483bp)  .. (Symbol{r0}_{1+ 1+ 1+ 1+ 0});
        \draw [->] (Mul{Pow{Symbol{N}+ NegativeOne{}}+ Symbol{r0}}_{1+ 1+ 1+ 1}) ..controls (359.7bp,102.76bp) and (365.12bp,92.007bp)  .. (Pow{Symbol{N}+ NegativeOne{}}_{1+ 1+ 1+ 1+ 1});
        \draw [->] (Pow{Symbol{N}+ NegativeOne{}}_{1+ 1+ 1+ 1+ 1}) ..controls (369.22bp,47.153bp) and (362.94bp,36.01bp)  .. (Symbol{N}_{1+ 1+ 1+ 1+ 1+ 0});
        \draw [->] (Pow{Symbol{N}+ NegativeOne{}}_{1+ 1+ 1+ 1+ 1}) ..controls (381.43bp,47.822bp) and (382.87bp,37.903bp)  .. (NegativeOne{}_{1+ 1+ 1+ 1+ 1+ 1});
        %
    \end{tikzpicture}
    }
    %        \vpace{50pt}

        \caption{Transformation of a four field equation into the Sympy expression tree}
        \caption*{In this figure the mapping from the inserted equation to the Sympy expression tree for the first step of Eq.~\eqref{eq:27} is shown. The condition $P(A \cap B) = P(A)P(B)$ is stored within the hashmap.
        Both expressions are mapped on to the same discrete probability $\frac{r_{1}}{N}*\frac{r_{2}}{N}$.}
        \label{fig:transformationfourfig}
    \end{figure}

\end{document}
